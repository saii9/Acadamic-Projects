% Generated by GrindEQ Word-to-LaTeX 
\documentclass{article} %%% use \documentstyle for old LaTeX compilers

\usepackage[english]{babel} %%% 'french', 'german', 'spanish', 'danish', etc.
\usepackage{amssymb}
\usepackage{amsmath}
\usepackage{txfonts}
\usepackage{mathdots}
\usepackage[classicReIm]{kpfonts}
\usepackage[pdftex]{graphicx} %%% use 'pdftex' instead of 'dvips' for PDF output

% You can include more LaTeX packages here 


\begin{document}

%\selectlanguage{english} %%% remove comment delimiter ('%') and select language if required


\noindent \textbf{Parallel and Distributed Systems}

\noindent \textbf{Assignment -- 5}

\noindent \textbf{SID : 004507888}

\noindent \textbf{}

\noindent Open MP program has been used to find the sum of the array. We have used the array sizes in 2 to the power. We have the readings for 2 to the power 8,10,12,14,16,18.To generate the reading run the shell script exe.sh. it will generate the data file for the aforementioned sizes. It can be done manually after compilation without using the shell script as below.

\noindent To Compile: gcc -fopenmp -lm -o main main.c

\noindent To Run       : ./main $<$power of the size of the array$>$ 

\noindent Each of the graphs obtained in the GNU plot as below. The GNU script file is named script.txt. to generate a plot using this script file, make sure the corresponding data file is present and issue the command 

\noindent ``gnuplot -c script.txt $<$power of the array$>$''

\noindent 

\noindent Plots :

fbox{\noindent \includegraphics*[scale = 0.5, bb=0 0 3in 3in, width=3in, height=3in, keepaspectratio=true]{out8.png}\includegraphics*[scale = 0.5, bb=0 0 3in 3in, width=3in, height=3in, keepaspectratio=true]{out10.png}}

\noindent \includegraphics*[scale = 0.5, bb=0 0 3in 3in, width=3in, height=3in, keepaspectratio=true]{out12.png}\includegraphics*[scale = 0.5, bb=0 0 3.21in 3in, width=3in, height=3in, keepaspectratio=true]{out14.png}

\noindent \includegraphics*[scale = 0.5, bb=0 0 3in 3in, width=3in, height=3in, keepaspectratio=true]{out16.png}\includegraphics*[scale = 0.5, bb=0 0 3in 3in, width=3in, height=3in, keepaspectratio=true]{out18.png}

\noindent 

\noindent From the above plots we can observe that for the lower order arrays the execution time increase as the number of the threads increases. This is because we are having a less computational problem distributed among many threads which makes the communication expensive. As we increase the size from 14 and above, we can see that as the number of threads increase the speedup increases.

\noindent 

\noindent 

\noindent 

\noindent The plots for the speed ups:

\noindent \includegraphics*[scale = 0.5,  bb=0 0 6.50in 4.87in, width=6.50in, height=4.87in, keepaspectratio=true]{speedup.png}

\noindent 

\noindent As we can see form the plot as the size of the array increase the communication cost goes down compared to the computation cost and speed up us achieved.

\noindent 


\end{document}

